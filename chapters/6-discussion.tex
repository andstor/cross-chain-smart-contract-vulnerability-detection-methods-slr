\chapter{Discussion}

In this \acrlong{slr}, efforts are made to systemize the state-of-the-art research on various blockchain platforms. A total of ? \todo{Fill in \# pages} papers (or online resources) available as of 2021 has been discussed and analyzed. From the findings presented in \cref{chap:results}, we can clearly see that the research of vulnerability analysis of other blockchain platforms than Ethereum is lacking. Even with deliberate efforts for covering other blockchain platforms, almost all research focuses exclusively on the Ethereum blockchain, with some minor exceptions being Hyperledger Fabric. The area of blockchain is immature. It is therefore expected that the popularity of Ethereum steals most of the research resources.

\section{Comparison with related work}
Compared to related surveys on the topic of smart contract vulnerability analysis and detection \cite{liu2019asurvey, huang2019smart, almakhour2020verification, he2020smart, vacca2021asystematic, kim2020analysis, peng2021security, wang2021security} (see \cref{chap:related-work}), this study covers a wider range of blockchain platforms. Where most surveys focus on the Ethereum blockchain, this study also focuses on other blockchain platforms. The closest related work to this study is \cite{wang2021security}. The authors provides a survey of security enhancement solutions on smart contracts. The survey addresses most static and dynamic analysis techniques, as well as some deep learning methods. Further, they state they provides a widened scope beyond just Ethereum. Our study covers provides an up-to-date \acrfull{slr} until 2021. Compared to \cite{wang2021security}, this study provides not only a more exhaustive list of vulnerability analysis and detection methods, but also a more detailed analysis of each tool and its capabilities. Further, this study also focus on the cross-applicability of the various detection methods.

\section{Threats to Validity}
The search strategy applied poses a likely threat to validity from missing out on or excluding relevant papers. Further, only one database was used for searching. To mitigate this, the search terms were iteratively improved. Further, an extensive snowballing process on references of the selected papers to identify related papers was conducted. During study selection, the researcher's subjective judgment could be a threat. The pre-defined review protocol was strictly followed in order to mitigate this. A widening of the search scope, as well as increasing the number of authors, could provide more relevant publications to complement and result in a more qualified analysis.
