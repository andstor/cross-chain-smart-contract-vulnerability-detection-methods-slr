\chapter{Introduction}

\begin{comment}
*Some sentences (two to three) of 
    Research background 
    Research motivation 
    Summarize the state of the art and state of the practice 
    To argue that your study is the world most interesting topic
    Research questions
    Research methods 
    Research results and main contributions
*Structure of the thesis
\end{comment}

The term Blockchain was popularized by Satoshi Nakamoto in 2008 with his publication of the article "Bitcoin: A Peer-to-Peer Electronic Cash System". A blockchain is a shared, immutable, digital ledger of transactions. Along with the launch of Bitcoin in 2009, this marks the birth of cryptocurrencies. In 2014, Ethereum extended the blockchain technology to include so-called \acrfullpl{sc}. A \acrshort{sc} is a program stored on a blockchain that runs when some predetermined conditions are met. This technology has paved the way for numerous new applications across a multitude of disciplines, including finance, health, education, and many more.

Blockchain has revolutionized the way in which we are able to share data in a secure, transparent, and traceable manner. This also applies to \acrshortpl{sc}. Once a smart contract is deployed, it cannot be changed. Due to most blockchains' monetary and anonymous nature, they pose as a desirable target for adversaries and manipulators \cite{atzei2017survey}. \acrshortpl{sc} often handle and store hundreds of millions of dollars worth of digital assets. This leads to deep concerns as to how to protect the assets from theft, fraud, and other forms of corruption. One of the most infamous blockchain attacks was the crowdfunding project \acrfull{dao} hack. This hack exploited a simple vulnerability in the contract language of Ethereum, resulting in an economic loss worth about 60 million dollars at the time \cite{atzei2017asurvey}. It is therefore imperative that all bugs and errors are pruned out before deployment, a difficult task.

There has been quite a lot of research devoted to vulnerability detection on the most popular blockchain platforms, mainly Ethereum, along with its smart contract language Solidity \cite{peng2021security, sing2020blockchain}. However, not much focus has been devoted to other, less popular blockchains \cite{chen2020asurvey, kim2020analysis}. As stated by \textcite{chen2020asurvey}, \enquote{The blockchain community keeps evolving very fast, and the most popular platform and language at one time can soon turn outdated.}. With the increasing popularity of blockchain technology, so does the need for security research.

In order to better secure the vast \acrshort{sc} ecosystem, one would need to find a way to detect and prevent these vulnerabilities across multiple blockchains (or cross-chain). In this \acrshort{slr}, cross-chain vulnerability detection is defined as a method for detecting vulnerabilities in smart contract code that can be applied for multiple blockchains. The area of \acrshortpl{sc} has attracted much academic interest. Still, a vast amount of the research on security methods of \acrshortpl{sc} is scattered across scientific papers, online resources, and forums. There is a need to structure and systemize the research to continue evolving the field. In particular, the methods which are transferrable to other blockchains are of particular importance.

The purpose of this document is to investigate the current state-of-the-art in vulnerability detection, in particular cross-chain vulnerability detection. Challenges and solutions for cross-chain vulnerability detection methods will be discussed, along with purposed future work. The research questions addressed in this \acrshort{slr} are:
\begin{itemize}
    \item RQ1 What are the current approaches for \acrlong{sc} vulnerability detection?
    \item RQ2 What is the current research on cross-chain \acrlong{sc} vulnerability detection?
\end{itemize}

Compared to other studies, this review will have a broader scope than just Ethereum by including other blockchain platforms. The specific contributions of this study are as follows:
\begin{itemize}
    \item Overview of state-of-the-art tools for analysis and detection of smart contract vulnerabilities.
    \item Wider scope than just Ethereum, including other blockchain platforms.
    \item Overview of the current research on cross-chain vulnerability detection.
    \item Identification of open issues, possible solutions to mitigate these issues, and future directions to advance the state of research in the domain.
\end{itemize}

The rest of this paper is organized as follows. \cref{chap:background} describes the background of the project. The studies related to this literature review are commented in \cref{chap:related-work}. \cref{chap:research-method} describes the methods used to detect vulnerabilities. \cref{chap:results} describes the results of the project, and \cref{chap:discussion} discuss the findings. Identified future work is presented in \cref{chap:future-work}. \cref{chap:conclusion} presents final remarks and concludes the paper.