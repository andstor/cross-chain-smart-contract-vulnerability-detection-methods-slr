\chapter{Introduction}

Over the years, several thesis templates for \LaTeX{} have been developed by different groups at NTNU. Typically, there have been local templates for given study programmes, or different templates for the different study levels – bachelor, master, and phd.\footnote{see, e.g., \url{https://github.com/COPCSE-NTNU/bachelor-thesis-NTNU} and \url{https://github.com/COPCSE-NTNU/master-theses-NTNU}}

Based on this experience, the CoPCSE\footnote{\url{https://www.ntnu.no/wiki/display/copcse/Community+of+Practice+in+Computer+Science+Education+Home}} is hereby offering a template that should in principle be applicable for theses at all study levels. It is closely based on the standard \LaTeX{} \texttt{report} document class as well as previous thesis templates. Since the central regulations for thesis design have been relaxed – at least for some of the historical university colleges now part of NTNU – the template has been simlified and put closer to the default \LaTeX{} look and feel.

The purpose of the present document is threefold. It should serve (i) as a description of the document class, (ii) as an example of how to use it, and (iii) as a thesis template.


Cross-chain vulnerability detection is a method for detecting vulnerabilities in smart contract code across multiple blockchains. Little efforts have been made to develop a cross-chain vulnerability detection framework. We discuss the challenges and opportunities of cross-chain vulnerability detection.

The purpose of this document is to investigate the current state-of-the-art in cross-chain vulnerability detection. We will discuss the challenges and opportunities of cross-chain vulnerability detection.


% Move to backround?
Once a \acrshort{sc} is deployed to a blockchain, any present vulnerabilities are permanent. There are some solutions for making \acrshortpl{sc} amendable, but this undermines the immutable property of \acrshortpl{sc}.