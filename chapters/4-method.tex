\chapter{Research Method}
\label{chap:research-method}

\section{Research Motivation}

Most prior works related to the detection and mitigation of software vulnerabilities have been focused on the software development life cycle. The software development process is a complex and iterative process. However, \acrshort{sc} requires a very different approach. Due to the immutable properties of \acrshort{sc}, all bugs and vulnerabilities needs to be removed before the code is put in production.

The area of cross-chain vulnerability analysis and detection has received limited attention. There are no papers primarily focused on cross-chain vulnerability analysis and detection. The research community is still in the early stages of the development of such a research topic. The need for a clear and systematic literature review of the current \acrshort{sc} vulnerability detection tools and methods that may be applicable for cross-chain is prominent. Through this \acrfull{slr}, an attempt to address this will be made by answering the research questions defined in \cref{sec:research-questions}.

\section{Research Questions}
\label{sec:research-questions}
The research questions addressed in this \acrshort{slr} are:
\begin{enumerate}[label=RQ\arabic*., leftmargin=1.5cm]
    \item What are the current approaches for \acrlong{sc} vulnerability detection?\\
    \item What is the current research on cross-chain \acrlong{sc} vulnerability detection?\\
    \item How to make \acrlong{sc} vulnerability detection possible cross-chain?\\
    \item What are the research gaps that need to be filled in future?
\end{enumerate}

\section{Research Method and Design}
The research method and design principles adopted by this \acrshort{slr} are based on the guidelines described by \textcite{kitchenham2007guidelines}. The process is divided into three phases.
\begin{enumerate}
    \item Identify the need for the review, prepare a proposal for the review, and develop the review protocol.
    \item Identify the research, select the studies, assess the quality, take notes and extract data, synthesize the data.
    \item search and selection of studies, extracting and synthesizing the data.
    \item Report the results of the review.
\end{enumerate}

\subsection{Locating Studies}
In the first step, we need to locate the studies that are related to \acrlong{sc} vulnerability detection. \acrfull{wos} was used as the main scientific database. The following search string for \acrshort{wos} was defined for this literature review:
\begin{quote}
    TS=(("smart contract" OR chaincode) AND (vulnerability OR bug) AND (detection OR analysis OR tool))
\end{quote}

Since we are also interested in the research of cross-chain \acrlong{sc} vulnerability detection, we include "chaincode" in the search string, as this is the equivalent of "smart contract" in Hyperledger Fabric \cite{hyperledger-fabric}.

\subsection{Study Selection and evaluation}
In order to assess the retrieved literature for it's eligibility, the inclusion and exclusion criteria listed in \cref{tab:inclusion-exclusion-criteria} were used. No time restrictions was set.


\begin{table}
    %\newcolumntype{Y}{>{\centering\arraybackslash}X}
    \def\arraystretch{1.5}
    \small
    \centering
    \caption{Existing ML-based smart contract vulnerability detection tools.}
    \label{tab:inclusion-exclusion-criteria}
    \begin{tabularx}{\textwidth}{lcX}
        \toprule
        & \textbf{ID} & \textbf{Selection criteria}\\
        \midrule
        \multirow{3}{*}{\rotatebox[origin=c]{90}{\textbf{Inclusion}}} & IC1 & Peer-reviewed research articles, conference proceedings papers, book chapters, and serials.\\
        & IC2 & Any time-frame\\
        \addlinespace[0.25cm]
        \midrule
        \addlinespace[0.25cm]
        \multirow{7}{*}{\rotatebox[origin=c]{90}{\textbf{Exclusion}}} & EC1 & Non english articles, missing abstracts, notes or editorials.\\
        & EC2 & Article is a copy or an older version of another publication already considered.\\
        & EC3 & Generic articles with title related to the blockchain technology and/or architecture.\\
        & EC4 & Article abstract addressing technical aspects of smart contract technology\\
        & EC5 & Article abstract addressing application (\glspl{dapp}) of smart contracts\\
        & EC6 & Article (full-text) not addressing vulnerability analysis or detection methods of smart contracts code.\\
        %& EC7 & Article (full-text) only addressing one specific type of vulnerability.\\
        \bottomrule
    \end{tabularx}
\end{table}

\subsection{Analysis and Synthesis}

The search resulted in 84 publications from \gls{wos}. The literature was then reduced in a series of steps. This process is visualized in \cref{fig:flowchart-search-selection-strategy}. In the identification stage, the literature was screened for any non-english articles, missing abstracts, or duplications. During screening, 21 articles were removed based on the title, and ? were removed based on the abstract. After full-text screening at the eligibility stage, ? articles were cut. The resulting literature was reduced to a total of ??? publications.

\begin{figure}
    \centering
    \noindent\begin{tikzpicture}
        [box/.style={rectangle,draw}]
        \small
        \definecolor{BlueLUH}{cmyk}{1.0,0.7,0,0}
        \tikzset{
            every node/.style={align=center,minimum width=0.7cm, node distance=5pt, very thick},
            arrow/.style={-{Triangle[width=\the\dimexpr1.8\pgflinewidth,length=\the\dimexpr0.8\pgflinewidth]}, shorten >=2pt,shorten <=2pt},
            order/.style={draw,circle},
            state/.style={draw=BlueLUH!90!white,rounded corners=5, minimum width=1cm},
            inclusion/.style={draw=black!60!green, text width=5cm, node distance=20pt and 5pt},
            exclusion/.style={draw=black!20!red, text width=3.5cm, node distance=20pt},
        }
    
        \node at (5, 1.5) {\large \textbf{Bibliographic database search}};
        
        \node[state,minimum height=3cm] (identification) at (0, -1) {\rotatebox{90}{Identification}};
        \node[state, below=of identification, minimum height=2.25cm] (screening) {\rotatebox{90}{Screening}};
        \node[state, below=of screening, minimum height=2cm] (eligibility) {\rotatebox{90}{Eligibility}};
        \node[state, below=of eligibility, minimum height=2cm] (inclusion) {\rotatebox{90}{Inclusion}};
        
        \node[order, left=10pt of identification] (stage1) {\large\textbf{1}};
        \node[order, left=10pt of screening] (stage2) {\large\textbf{2}};
        \node[order, left=10pt of eligibility] (stage3) {\large\textbf{3}};
        \node[order, left=10pt of inclusion] (stage4) {\large\textbf{4}};
        
        \node[inclusion,right=of identification.north east, anchor=north west, minimum height=3cm] (retrieved) {Records retrieved (n=125). Additional articles included through bibliographic trail search and reference list (n=??).};
        \node[inclusion,right=of screening.north east, anchor=north west,minimum height=2.25cm] (title-abstract) {Title and abstract screening (n=103)};
        \node[inclusion,right=of eligibility, minimum height=2cm] (full-text) {Full-text articles assessed for eligibility (n=?)};
        \node[inclusion,right=of inclusion.north east, anchor=north west, minimum height=2cm] (included) {Total number of articles included (n=?)};
        %\node[inclusion,below=of included.south west, anchor=north west, minimum height=1cm] (total) {Total records included in qualitative analysis (n=195): 206 articles and 54 reports};
        
        \node[exclusion,right=of retrieved] (exclusion) {Exclusion of non-English language literature, missing abstracts, duplications (n=1).};
        \node[exclusion,right=of title-abstract.north east, anchor=north west] (title) {Articles excluded based on Title (n=21)};
        \node[exclusion,below=of title.west, anchor=north west] (abstract) {Articles excluded based on Abstract (n=?)};
        \node[exclusion,right=of full-text] (full-text-ex) {Full-text articles excluded (n=?)};
        
        \draw[line width=7pt,arrow] (retrieved) -- (exclusion);
        \draw[line width=7pt,arrow] (title-abstract.east) +(0,1.75em) coordinate (b1) -- (title.west |- b1);
        \draw[line width=7pt,arrow] (title-abstract.east) +(0,-1.75em) coordinate (b1) -- (abstract.west |- b1);s
        %\draw[line width=7pt,arrow] (retrieved) -- (title-abstract);
        %\draw[line width=7pt,arrow] (title-abstract) -- (full-text);
        %\draw[line width=7pt,arrow] (full-text) -- (included);
        \draw[line width=7pt,arrow] (full-text) -- (full-text-ex);
        %\draw[line width=7pt,arrow] (included.south) coordinate (a1) -- (total.north west -| a1);
        %\draw let \p13 = ($(retrieved.north east) + (0,0.5)$) in node {\x13 \y13};
    \end{tikzpicture}
    \caption{Flowchart of search and selection \acrshort{slr} strategy.}
    \label{fig:flowchart-search-selection-strategy}
    \end{figure}