\chapter{Research Method}

Research projects should always be based on previous research on the same and/or related topics. This should be described as a background to the thesis with adequate bibliographical references. If the material needed is too voluminous to fit nicely in the review part of the introduction, it can be presented in a separate background chapter.

\section{Research Motivation}

Most prior works related to the detection and mitigation software vulnerabilities have been focused on the software development life cycle. The software development process is a complex and iterative process. However, \acrshort{sc} requires a very different approach. Due to the immutable properties of \acrshort{sc}, all bugs and vulnerabilities needs to be removed before the code is put in production.

\section{Research Questions}

RQ1: What is the top \acrlong{sc} vulnerabilities?
RQ2: What are the current approaches for \acrlong{sc} vulnerability detection?
RQ3: What are the current defenses against \acrlong{sc} vulnerabilities?
RQ4: What is the current research on cross-chain \acrlong{sc} vulnerability detection?
RQ5: How to make \acrlong{sc} vulnerability detecting possible cross-chain?