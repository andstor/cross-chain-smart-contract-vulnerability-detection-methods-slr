\chapter{Research Method}
\label{chap:research-method}
This chapter presents the research method used in this \acrfull{slr}. Firstly, the research motivation is presented, followed by the research questions defined for this \acrshort{slr}. Finally, the research method and design is explained.

\section{Research Motivation}

Most prior works related to the detection and mitigation of software vulnerabilities have been focused on the software development life cycle. The software development process is a complex and iterative process. However, \acrshort{sc} requires a very different approach. Due to the immutable properties of \acrshort{sc}, all bugs and vulnerabilities needs to be removed before the code is put in production.


%The area of cross-chain vulnerability analysis and detection has received limited attention, as can be seen from \cref{tab:related-work}. There are no papers primarily focused on cross-chain vulnerability analysis and detection. The research community is still in the early stages of the development of such a research topic. The need for a clear and systematic literature review of the current \acrshort{sc} vulnerability detection tools and methods that may be applicable for cross-chain is prominent. In order to catch-up with the increasing blockchain development of new blockchain plattforms, the  Not only would this accelerate the development process of security tools for other chains than Ethereum, but also  to make the detection approach transferable, but also .....???? Through this \acrfull{slr}, an attempt to address this will be made by answering the research questions defined in \cref{sec:research-questions}.

The blockchain technology has surged in popularity over last years, giving birth to several new blockchain platforms (see \cref{tab:blockchain-platforms}). As the list of new blockchains is expanding, so does the number of vulnerabilities, increasing the attack surface of the blockchain ecosystem \cite{technative2018gogan}. According to \textcite{huang2019smart}, \acrshort{sc} vulnerabilities may arise from the \acrshort{sc} language used, the specific features of the blockchain, or from misunderstanding of common practices. Using tools to detect \acrshort{sc} vulnerabilities is a key step on the road of securing \acrshortpl{sc}. However, the research area on \acrshort{sc} security on other blockchain platforms than Ethereum (and to some degree Hyperledger Fabric) is still in its infancy. The need for a clear and systematic literature review of the current \acrshort{sc} vulnerability detection tools and methods, regardless of platform, is therefore prominent. Further, in order to speed up the development of this research area, an investigation of tools and methods that may be applicable for cross-chain is needed. There are currently no papers primarily focused on cross-chain vulnerability analysis and detection. By making the detection approach transferable, this would accelerate the development process of vulnerability detection tools and methods. Through this \acrshort{slr}, an attempt to address these issues will be made by answering the research questions defined in \cref{sec:research-questions}.

\section{Research Questions}
\label{sec:research-questions}
The research questions addressed in this \acrshort{slr} are:
\begin{enumerate}[label=RQ\arabic*., leftmargin=1.5cm]
    \item What are the current approaches for \acrlong{sc} vulnerability detection?
    \item What is the current research on cross-chain \acrlong{sc} vulnerability detection?
\end{enumerate}

\section{Research Method and Design}
The research method and design principles adopted by this \acrshort{slr} are based on the guidelines described by \textcite{kitchenham2007guidelines}. The process is divided into three phases.
\begin{enumerate}
    \item Identify the need for the review, prepare a proposal for the review, and develop the review protocol.
    \item Identify the research, select the studies, assess the quality, extract and synthesize the data.
    \item Report the results of the review.
\end{enumerate}

\subsection{Locating Studies}
In the first step, the studies related to \acrfull{sc} vulnerability detection were needed to locate. \acrfull{wos} was used as the main scientific database. The following search string for \acrshort{wos} was defined for this literature review:
\begin{quote}
    TS=(("smart contract" OR chaincode) AND (vulnerability OR bug) AND (detection OR analysis OR tool))
\end{quote}

Since this study is also interested in the research of cross-chain \acrlong{sc} vulnerability detection, "\gls{chaincode}" is included in the search string, as this is the equivalent of "\acrlong{sc}" in Hyperledger Fabric \cite{hyperledger-fabric}.

\subsection{Study Selection}
In order to assess the retrieved literature for its eligibility, the inclusion and exclusion criteria listed in \cref{tab:inclusion-exclusion-criteria} were used. No time restrictions were set.

The search resulted in 125 publications from \gls{wos}. The literature was then reduced in a series of steps. This process is visualized in \cref{fig:flowchart-search-selection-strategy}. In the identification stage, the literature was screened for any non-English articles, missing abstracts, or duplications. 3 parers were removed. During the screening, 21 articles were removed based on the title, and 39 were removed based on the abstract. After full-text screening at the eligibility stage, 38 articles were cut. Backwards snowballing was then applied by scanning the references of the selected papers. This resulted in 16 additional papers. In this \acrshort{slr}, the resulting literature was reduced to a total of 40 primary studies.

\begin{table}
    %\newcolumntype{Y}{>{\centering\arraybackslash}X}
    \def\arraystretch{1.5}
    \small
    \centering
    \caption{Inclusion and exclusion criteria.}
    \label{tab:inclusion-exclusion-criteria}
    \begin{tabularx}{\textwidth}{lcX}
        \toprule
        & \textbf{ID} & \textbf{Selection criteria}\\
        \midrule
        \multirow{3}{*}{\rotatebox[origin=c]{90}{\textbf{Inclusion}}} & IC1 & Peer-reviewed research articles, conference proceedings papers, book chapters, and serials.\\
        & IC2 & Any time-frame\\
        \addlinespace[0.25cm]
        \midrule
        \addlinespace[0.25cm]
        \multirow{4}{*}{\rotatebox[origin=c]{90}{\textbf{Exclusion}}} & EC1 & Non english articles, missing abstracts, notes or editorials.\\
        & EC2 & Article is a copy or an older version of another publication already considered.\\
        & EC3 & Article not addressing vulnerability analysis or detection methods of smart contracts code.\\
        \bottomrule
    \end{tabularx}
\end{table}

\begin{figure}
    \centering
    \noindent\begin{tikzpicture}
        [box/.style={rectangle,draw}]
        \small
        \definecolor{BlueLUH}{cmyk}{1.0,0.7,0,0}
        \tikzset{
            every node/.style={align=center,minimum width=0.7cm, node distance=5pt, very thick},
            arrow/.style={-{Triangle[width=\the\dimexpr1.8\pgflinewidth,length=\the\dimexpr0.8\pgflinewidth]}, shorten >=2pt,shorten <=2pt},
            order/.style={draw,circle},
            state/.style={draw=BlueLUH!90!white,rounded corners=5, minimum width=1cm, node distance=20pt},
            inclusion/.style={draw=black!60!green, text width=5cm, node distance=20pt and 5pt},
            exclusion/.style={draw=black!20!red, text width=3.5cm, node distance=23pt and 5pt},
        }
            
        \node[state,minimum height=3cm] (identification) at (0, 0) {\rotatebox{90}{Identification}};
        \node[state, below=of identification, minimum height=2.25cm] (screening) {\rotatebox{90}{Screening}};
        \node[state, below=of screening, minimum height=2cm] (eligibility) {\rotatebox{90}{Eligibility}};
        \node[state, below=of eligibility, minimum height=2.5cm] (inclusion) {\rotatebox{90}{Snowballing}};
        
        \node[order, left=10pt of identification] (stage1) {\large\textbf{1}};
        \node[order, left=10pt of screening] (stage2) {\large\textbf{2}};
        \node[order, left=10pt of eligibility] (stage3) {\large\textbf{3}};
        \node[order, left=10pt of inclusion] (stage4) {\large\textbf{4}};
        
        \node[inclusion,right=of identification.north east, anchor=north west, minimum height=3cm] (retrieved) {Records retrieved (n=125).};
        \node[inclusion,right=of screening.north east, anchor=north west,minimum height=2.25cm] (title-abstract) {Title and abstract screening (n=62)};
        \node[inclusion,right=of eligibility, minimum height=2cm] (full-text) {Full-text articles assessed for eligibility (n=24)};
        \node[inclusion,right=of inclusion.north east, anchor=north west, minimum height=2.5cm] (snowballing) {Additional 16 articles included through snowballing (n=40)};
        
        \node[exclusion,right=of retrieved] (exclusion) {Exclusion of non-English language literature, missing abstracts, duplications (n=3).};
        \node[exclusion,right=of title-abstract.north east, anchor=north west] (title) {Articles excluded based on title (n=21)};
        \node[exclusion,below=of title.west, anchor=north west] (abstract) {Articles excluded based on abstract (n=39)};
        \node[exclusion,right=of full-text] (full-text-ex) {Full-text articles excluded (n=38)};
        
        \draw[line width=7pt,arrow] (retrieved) -- (title-abstract);
        \draw[line width=7pt,arrow] (title-abstract) -- (full-text);
        \draw[line width=7pt,arrow] (full-text) -- (snowballing);
        
        %\draw[line width=7pt,arrow] (retrieved) -- (exclusion);
        %\draw[line width=7pt,arrow] (title-abstract.east) +(0,1.75em) coordinate (b1) -- (title.west |- b1);
        %\draw[line width=7pt,arrow] (title-abstract.east) +(0,-1.75em) coordinate (b1) -- (abstract.west |- b1);s
        %\draw[line width=7pt,arrow] (full-text) -- (full-text-ex);
        
        %\draw[line width=7pt,arrow] (retrieved) -- (title-abstract);
        %\draw[line width=7pt,arrow] (title-abstract) -- (full-text);
        %\draw[line width=7pt,arrow] (full-text) -- (included);
        %\draw[line width=7pt,arrow] (included.south) coordinate (a1) -- (total.north west -| a1);
        %\draw let \p13 = ($(retrieved.north east) + (0,0.5)$) in node {\x13 \y13};
    \end{tikzpicture}
    \caption{Flowchart of search and selection \acrshort{slr} strategy.}
    \label{fig:flowchart-search-selection-strategy}
    \end{figure}