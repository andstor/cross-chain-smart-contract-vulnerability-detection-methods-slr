\chapter{Research Method}
\label{chap:research-method}



\section{Research Motivation}

Most prior works related to the detection and mitigation software vulnerabilities have been focused on the software development life cycle. The software development process is a complex and iterative process. However, \acrshort{sc} requires a very different approach. Due to the immutable properties of \acrshort{sc}, all bugs and vulnerabilities needs to be removed before the code is put in production.

\section{Research Questions}

RQ1: What are the current approaches for \acrlong{sc} vulnerability detection?\\
RQ2: What is the current research on cross-chain \acrlong{sc} vulnerability detection?\\
RQ3: How to make \acrlong{sc} vulnerability detecting possible cross-chain?\\

\section{Research Method and Design}
The research method and design principles adopted in this \acrfull{slr} are based on the following principles by ....:\\


\subsection{Locating Studies}
In the first step, we need to locate the studies that are related to \acrlong{sc} vulnerability detection. \acrfull{wos} was used as the main scientific database. The following search string for \acrshort{wos} was defined for this literature review:
\begin{quote}
    TS=("smart contract" vulnerability AND (detection OR analysis))
\end{quote}
No time restrictions was set. The search results were exported to a \gls{bibtex} file and processed locally.
\todo{Include articles retrievedby sources.}

\subsection{Study Selection and evaluation}
In order to assess the retrieved literature for it's eligibility, the inclusion and exclusion criteria listed in \cref{tab:inclusion-exclusion-criteria} were used.

\begin{table}
    %\newcolumntype{Y}{>{\centering\arraybackslash}X}
    \def\arraystretch{1.5}
    \small
    \centering
    \caption{Existing ML-based smart contract vulnerability detection tools.}
    \label{tab:inclusion-exclusion-criteria}
    \begin{tabularx}{\textwidth}{*{3}{X}}
        \toprule
        & \textbf{Stage} & \textbf{Selection criteria} \\
        \midrule
        Inclusion & Identification & Non english articles, missing abstracts, notes or editorials.\\
        & Screening & col3\\
        \hline
        Exclusion & Eligibility & col3\\
        & Inclusion & Inclusion\\
        & col2 & col3\\
        & col2 & col3\\
        \bottomrule
    \end{tabularx}
\end{table}

\subsection{Analysis and Synthesis}
The search resulted in 84 publications from \gls{wos}. The literature was then reduced in a series of steps. This process is visualized in \cref{fig:flowchart-search-selection-strategy}. In the identification stage, the literature was screened for any non-english articles, missing abstracts, notes or editorials. During screening, ? articles were removed based on the title, and ? were removed based on the abstract. After full-text screening at the eligibility stage, ? articles were cut. The resulting literature was reduced to a total of ??? publications.

\begin{figure}
\centering
\noindent\begin{tikzpicture}
    [box/.style={rectangle,draw}]
    \small
    \definecolor{BlueLUH}{cmyk}{1.0,0.7,0,0}
    \tikzset{
        every node/.style={align=center,minimum width=0.7cm, node distance=20pt, very thick},
        arrow/.style={-{Triangle[width=\the\dimexpr1.8\pgflinewidth,length=\the\dimexpr0.8\pgflinewidth]}, shorten >=2pt,shorten <=2pt},
        order/.style={draw,circle},
        state/.style={draw=BlueLUH!90!white,rounded corners=5, minimum width=1cm},
        inclusion/.style={draw=black!60!green, text width=5cm, node distance=20pt and 5pt},
        exclusion/.style={draw=black!20!red, text width=3.5cm, node distance=20pt},
    }

    \node at (5, 1.5) {\large \textbf{Bibliographic database search}};
    
    \node[state,minimum height=3cm] (identification) at (0, -1) {\rotatebox{90}{Identification}};
    \node[state, below=of identification, minimum height=2.25cm] (screening) {\rotatebox{90}{Screening}};
    \node[state, below=of screening, minimum height=2cm] (eligibility) {\rotatebox{90}{Eligibility}};
    \node[state, below=of eligibility, minimum height=2cm] (inclusion) {\rotatebox{90}{Inclusion}};
    
    \node[order, left=10pt of identification] (stage1) {\large\textbf{1}};
    \node[order, left=10pt of screening] (stage2) {\large\textbf{2}};
    \node[order, left=10pt of eligibility] (stage3) {\large\textbf{3}};
    \node[order, left=10pt of inclusion] (stage4) {\large\textbf{4}};
    
    \node[inclusion,right=of identification.north east, anchor=north west, minimum height=3cm] (retrieved) {Records retrieved (n=84). Additional articles included through bibliographic trail search and reference list (n=??).};
    \node[inclusion,right=of screening.north east, anchor=north west,minimum height=2.25cm] (title-abstract) {Title and abstract screening (n=?)};
    \node[inclusion,right=of eligibility, minimum height=2cm] (full-text) {Full-text articles assessed for eligibility (n=?)};
    \node[inclusion,right=of inclusion.north east, anchor=north west, minimum height=2cm] (included) {Total number of articles included (n=?)};
    %\node[inclusion,below=of included.south west, anchor=north west, minimum height=1cm] (total) {Total records included in qualitative analysis (n=195): 206 articles and 54 reports};
    
    \node[exclusion,right=of retrieved] (exclusion) {Exclusion of non-English language literature, missing abstracts, notes or editorials (n=?).};
    \node[exclusion,right=of title-abstract.north east, anchor=north west] (title) {Articles excluded based on Title (n=?)};
    \node[exclusion,below=of title.west, anchor=north west] (abstract) {Articles excluded based on Abstract (n=?)};
    \node[exclusion,right=of full-text] (full-text-ex) {Full-text articles excluded (n=?)};
    
    \draw[line width=7pt,arrow] (retrieved) -- (exclusion);
    \draw[line width=7pt,arrow] (title-abstract.east) +(0,1.75em) coordinate (b1) -- (title.west |- b1);
    \draw[line width=7pt,arrow] (title-abstract.east) +(0,-1.75em) coordinate (b1) -- (abstract.west |- b1);s
    \draw[line width=7pt,arrow] (retrieved) -- (title-abstract);
    \draw[line width=7pt,arrow] (title-abstract) -- (full-text);
    \draw[line width=7pt,arrow] (full-text) -- (included);
    \draw[line width=7pt,arrow] (full-text) -- (full-text-ex);
    %\draw[line width=7pt,arrow] (included.south) coordinate (a1) -- (total.north west -| a1);
    %\draw let \p13 = ($(retrieved.north east) + (0,0.5)$) in node {\x13 \y13};
\end{tikzpicture}
\caption{Flowchart of search and selection \acrshort{slr} strategy.}
\label{fig:flowchart-search-selection-strategy}
\end{figure}