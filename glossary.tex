
\usepackage{xparse}
\DeclareDocumentCommand{\newdualentry}{ O{} O{} m m m m } {
    \newglossaryentry{gls-#3}{name={#5},text={#5\glsadd{#3}},
        description={#6},#1
    }
    \newacronym[see={[Glossary:]{gls-#3}},name={#4},#2]{#3}{#4\glsadd{gls-#3}}{#5}
} % as per http://en.wikibooks.org/wiki/LaTeX/Glossary

\makeglossaries % Prepare for adding glossary entries

% ----------------------
% ----- Glossaries -----
% ----------------------

\newglossaryentry{solidity}
{
    name=Solidity,
    description={Solidity is a \acrlong{sc} language developed for Ethereum}
}

\newglossaryentry{gas}
{
    name=gas,
    description={Gas in crypto refers to the computational effort required to execute operations. This is normally paid in the blockchain's platform cryptocurrency}
}

\newglossaryentry{llvm}
{
    name=LLVM,
    description={A compiler toolchain providing a complete infrastructure for creating compiler frontends and backends}
}

\newglossaryentry{llvmir}
{
    name=LLVM-IR,
    description={LLVM-IR is the intermediate representation of the LLVM compiler toolchain}
}

\newglossaryentry{f-star}
{
    name=F*,
    description={General-purpose functional programming language with effects aimed at program verification}
}

\newglossaryentry{bibtex}
{
    name=BibTeX,
    description={File format for storing bibliographic information}
}

\newglossaryentry{bytecode}
{
    name=bytecode,
    description={Bytecode is computer object code that is processed by a program (VM)s}
}

\newglossaryentry{dapp}
{
    name=dapp,
    description={A decentralized application is a computer application that runs on a decentralized computing system}
}
\newglossaryentry{smt-solver}
{
    name=SMT solver,
    description={A tool for solving a \acrlong{smt} problem for a practical subset of inputs}
}

\newglossaryentry{z3}
{
    name=Z3,
    description={A theorem prover (SMT solver) from Microsoft Research}
}

\newglossaryentry{oracle}
{
    name=oracle,
    description={Entities that connect blockchains to external systems, enabling smart contracts to execute based upon inputs and outputs from the real world}
}

\newglossaryentry{deep-learning}
{
    name=deep learning,
    description={Deep learning is a field of computer science, artificial intelligence, and statistics that deals with the learning of representations, particularly for purposes of machine learning and natural language processing}
}


% --------------------
% ----- Acronyms -----
% --------------------

\newacronym{sc}{SC}{Smart Contract}
\newacronym{dao}{DAO}{Decentralized Autonomous Organization}
\newacronym{ml}{ML}{Machine Learning}
\newacronym{slr}{SLR}{Systematic Literature Review}
\newacronym{cfg}{CFG}{Control Flow Graph}
\newacronym{ast}{AST}{Abstract Syntax Tree}
\newacronym{abi}{ABI}{Application Binary Interface}
\newacronym{rf}{RF}{Random Forest}
\newacronym{svm}{SVM}{Support Vector Machine}
\newacronym{dnn}{DNN}{Deep Neural Network}
\newacronym{cnn}{CNN}{Convolutional Neural Network}
\newacronym{knn}{KNN}{K-Nearest Neighbor}
\newacronym{sgd}{SGD}{Stochastic Gradient Descent}
\newacronym{gan}{GAN}{Generative Adversarial Network}
\newacronym{lstm}{LSTM}{Long Short-Term Memory}
\newacronym{att-blstm}{Att-BLSTM}{Attention-Based Bidirectional Long Short-Term Memory}
\newacronym{tfidf}{TFIDF}{Term Frequency–Inverse Document Frequency}
\newacronym{iot}{IoT}{Internet of Things}
\newacronym{pdg}{PDG}{Program Dependency Graph}
\newacronym{ssg}{SSG}{Symbolic Semantic Graphs}


% -----------------------
% ----- Dual entries -----
% -----------------------

\newdualentry{nft} % label
  {NFT}            % abbreviation
  {Non Fungible Tokens}  % long form
  {A type of token that is unique} % description

  \newdualentry{ir} % label
  {IR}            % abbreviation
  {Intermediate Representation}  % long form
  {A representation for use as an intermediate step} % description
  
  \newdualentry{wos} % label
  {WoS}            % abbreviation
  {Web of Science}  % long form
  {Scientific citation database} % description
  
  \newdualentry{smt} % label
  {SMT}            % abbreviation
  {Satisfiability Modulo Theories}  % long form
  {The problem of determining whether a mathematical formula is satisfiable} % description