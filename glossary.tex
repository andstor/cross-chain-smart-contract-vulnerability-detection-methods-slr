
\usepackage{xparse}
\DeclareDocumentCommand{\newdualentry}{ O{} O{} m m m m } {
    \newglossaryentry{gls-#3}{name={#5},text={#5\glsadd{#3}},
        description={#6},#1
    }
    \newacronym[see={[Glossary:]{gls-#3}},name={#4},#2]{#3}{#4\glsadd{gls-#3}}{#5}
} % as per http://en.wikibooks.org/wiki/LaTeX/Glossary

\makeglossaries % Prepare for adding glossary entries

% ----------------------
% ----- Glossaries -----
% ----------------------

\newglossaryentry{solidity}
{
    name=Solidity,
    description={Solidity is a \acrlong{sc} language developed for Ethereum}
}

\newglossaryentry{gas}
{
    name=gas,
    description={Gas in crypto refers to the computational effort required to execute operations. This is normally paid in the blockchain's plattform cryptocurrency.}
}

\newglossaryentry{llvm}
{
    name=LLVM,
    description={A compiler toolchain providing a complete infrastructure for creating compiler frontends and backends.}
}

\newglossaryentry{llvmir}
{
    name=LLVM-IR,
    description={LLVM-IR is the intermediate representation of the LLVM compiler toolchain}
}

% --------------------
% ----- Acronyms -----
% --------------------

\newacronym{sc}{SC}{Smart Contract}
\newacronym{ml}{ML}{Machine Learning}


% -----------------------
% ----- Dual entries -----
% -----------------------

\newdualentry{nft} % label
  {NFT}            % abbreviation
  {Non Fungible Tokens}  % long form
  {A type of token that is unique} % description

  \newdualentry{ir} % label
  {IR}            % abbreviation
  {Intermediate Representation}  % long form
  {A representation for use as an intermediate step} % description